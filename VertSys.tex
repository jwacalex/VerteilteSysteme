\documentclass[pagesize,11pt,a4paper]{scrartcl}
\usepackage[T1]{fontenc}
\usepackage[utf8]{inputenc}
\usepackage[ngerman]{babel}



\begin{document}

\paragraph*{Verteiltes System}
	Ansammlung unabhängiger Computer, die wie ein einzelnes System erscheinen; 
	Menge interagierender Prozesse, über Nachrichten kommunizierend;

\paragraph*{Ziele für Verteilte Systeme}
	Gemeinsame Ressourcennutzung;
	Transparenz (Zugriff, Ort, Migration, Relokation, Replikation, Nebenläufigkeit);
	Offenheit (Spezifiziert u. Vollständigkeit. Folge: Interoperabel, Portabel) (sei konservativ in dem, was Du sendest, sei tolerant in dem, was Du empfängst);
	Skalierbarkeit (Bottleneck, Single Point of Failure, etc) (durch Async, Aufteilung, etc);
	
\paragraph*{Cloud}
	Abstraktion: logische / physikalische Transparenz; 

\paragraph{Architekturen}
	Geschichtet:ISO/OSI, Multi-Tier, Drei-Tier (Dispatch-Worker), Middleware;
	Ereignisbasiert: Push/Subscribe
	Objektbasiert: RMI/RPC;
	Zentral: Client/Server;
	Dezentral: Overlay: Mutlicast auf Anwendungsbene, P2P (Superpeer-Konzept);
	Hybridformen
	
\paragraph{Prozess vs Thread}
	Thread: leichtgewichter Prozess;
	LWP: Vergleiche mit Worker in Java;

\paragraph*{Mutex}
	Lösung: Semaphoren, Monitore, Locks, Warten;
	Konkurierender Zugriff (Deadlock, Starvation);
	Lösungen: 
		Tokenbasierte (Probleme: Tokenverlust, nur ein Token), 
		Koordinator (Loadbalanic, etc, aber single point of failure, bottleneck),
		Dezentraler Koordinator: (Jeder Prozess Kordiniert, Problem: ein Prozess stürzt ab) - 
		Kordinatorbestimmung: 
			Bully - Wahlnachricht an Prozesse höhere Nummern|Wenn kein Prozess antwortet, übernimmt Prozess| ansonsten rekursiv weiter,
			Ring: Nachfolger Übernimmt (oder nächster) (Nachricht wird einmal rumgereicht)

\paragraph*{ACID}
	Transaktionskonsitenz:
		Atomic Consistent Isolation Durability;
	Transaktionsmonitor überwacht

\paragraph*{ISO-Modell}
	Physical, Data, Network, Transport, Session, Presentation, Application
	
\paragraph*{Stubs}
	Stumpf (Anknüpfungspunkt);
	Client-Stub: ersetzt lokalen Prozeduraufruf;
	Parameter an Server, warten, etc;
	Server-Stub: Operation, Datenverarbeiten;

\paragraph*{RPC}
	sync. Kontrollfluss/Datenuebergabe, Prozeduraufruf, unterschiedliche Adressräume, schmaler Kanal;
	parallelität durch asynch;
	
\paragraph*{Berkeley Sockets}
	Schnittstelle f. nachrichtenorientierte, fluechtige Kommunikation

\paragraph*{MPI}
	\\m/

\paragraph*{Marshalling}
	Umwandeln von strukturieren Daten in Übertragungsformat
	
\paragraph*{Serialisierung}
	Sturkturierte Daten in sequenzielles Format (Speicherung, implementation von marschalling)
	
\paragraph*{Kommunikation}
	persistent: Speicherung in Mittleware bis Zustellung;
	transient (flüchtig): Anwendungsausführungszeit

\paragraph{Chord}
	Entität mit Schlüssel k -> Konten mit Bezeichner id >= k (bzw. k <= id). Nachfolger von k;
	Suche: Knoten betrachtet Vogänger und Nachfolger, wenn Vorgängern < k < Konten -> Knoten zuständig, sonst weiterleitung;
	Bei Fingertabelle: Ordne Schlüssel bei einträgen ein Eintrag i < k < Eintrag i+1 => Weiterleitung Eintrag 1 -> Sonst k an Knoten;
	Hinzukommen/Verlassen: Einfach in Ring einbauen; Aktualisierung der Fingertabellen: Prüfen ob man Vorgänger des Nachfolgers ist ->
		Falls ja: alles ok -
		Fall nein: neuer Knoten -> Knoten < neuer Knoten <= Nachfolger d. Knotens. Anpassung der ersten Position der Fingertabelle. Für jeden weiteren Eintrag, analog. im Nachfolger ebenfalls durch den Vorgänger-

\paragraph*{Cristians Algo}
	T1: Absenden, T2: Eintreffen Zeitserver, T3: Absenden Antwort Zeitserver; T4: Eintreffen Rechner;
	T2 -T1: Laufzeit Anfrage; T4-T3: Laufzeit Antwort;
	Laufzeit in eine Richtung -> Mittel aus beiden Anfragen;
	Abweichung der Uhr Aufaddieren: ((T2-T1)+(T3-T4)) / 2;
	NIEMALS RÜCKLAUFEN DER UHR;

\paragraph*{NTP}
	Implementiert Cristian;
	Uhr weiss wie gut sie ist;

\paragraph*{Berkely}
	Aktiver Zeitserver; Fragt Systeme an; Bilden Durschnittszeit; Angleichen;
	Nur möglich wenn gemeinsame Zeit innerhalb System notwendig;

\paragraph*{In Drahtlosen Systeme}
	Kommunkation teuer; Reference Broadcast (RBS);

\paragraph*{Logische Uhren}
	Keine Notwendigkeit bei bei fehlender Wechselwirkung, Nur Ereignisreihenfolge wichtig;
	Happens-Before-Relation: Alle Prozesse stimmen ein, dass A vor B stattfand.
		Beispiel: A und B nacheinander im gleichen Prozess -
		oder A sendet Nachricht, B empfängt diese Nachricht -> A vor B;
	sind transitiv, Zeitwerte von A ist kleiner als von B;
	Uhr geht immer vorwärts (Nur Addition)

\paragraph*{Lamportuhr}
	Vor Ereignis: Erhöhen des Zählers;
	Bei Transaktion: Zeitstempel (Zähler mitsenden);
	Bei Empfang: höheren Wert aus (lokalem Zähler und Zeitstempel) nehmen Danach Zähler erhöhen;
	Keine zwei Ereignisse mit gleicher Zeit.

\paragraph*{Vectorclock}
	Wie Lamportuhr, jedoch Zähler für jeden Prozess;

\paragraph{Replikation}
	Caching, Datenverteilung/Duplizierung

\paragraph{Konsistenz}
	stufenlos: Conit (kleinste konsistente Einheit (Quasiatomar));
	sequenztiell: Paralleler Datenzugriff (kein falsches lesen);
	kausal: Schreibvorgänge in Kausaler Beziehung;
	eventual consistency: schwache Konsistenz, Replikate nach und nach aktualisieren;
	monotones Lesen: wenn prozess datensatz liesst -> immer x oder aktuelleren Zurückgeben;
	read your writes: schreiboperation wird immer vor leseoperation abgeschlossen;
	write follows read: wenn nach lesen geschrieben wird, hat operation aktuellen oder selben wert f. x
	

\end{document}